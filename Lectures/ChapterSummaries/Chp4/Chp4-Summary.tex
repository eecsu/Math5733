%Title: Chp Lecture 03
%Started:
%Updated:
\documentclass{amsart}
\usepackage{amssymb,latexsym,amsmath}
\usepackage{graphicx}
\usepackage{epsfig}
\usepackage{color}
\usepackage{enumerate}
%-----------------------------------------------------------------
\vfuzz2pt % Don't report over-full v-boxes if over-edge is small
\hfuzz2pt % Don't report over-full h-boxes if over-edge is small
% THEOREMS -------------------------------------------------------
\theoremstyle{plain}
\newtheorem{thm}{Theorem}
\newtheorem{cor}{Corollary}
\newtheorem{lem}{Lemma}
\newtheorem{prop}{Proposition}
\theoremstyle{definition}
\newtheorem{defn}{Definition}
\theoremstyle{remark}
\newtheorem{rem}{Remark}
\theoremstyle{definition}
\newtheorem{ex}{Example}
\numberwithin{equation}{section}
\newtheorem{prob}{Problem}
\numberwithin{equation}{section}
% Colors-----------------------------------------------------------
\definecolor{Green}{rgb}{0,.5,0}
%use for definitions
\definecolor{Red}{rgb}{.8,.2,0}
%use for emphasis
\definecolor{Yellow}{rgb}{.6,.6,.1}
%use for part titles
\definecolor{Cyan}{rgb}{.2,.6,.7}
%use for comments
\definecolor{Purple}{rgb}{.4,0,1}
%use for examples
\definecolor{deepred}{rgb}{.53,.29,.24}
%use for important points
\definecolor{Black}{rgb}{0,0,0}
%use for washout
\definecolor{Grey}{rgb}{.45,.45,.45}
% use for theorems
\newcommand{\tred}[1]{\textcolor{Red}{#1}}
\newcommand{\tgreen}[1]{\textcolor{Green}{#1}}
\newcommand{\tcyan}[1]{\textcolor{Cyan}{#1}}
\newcommand{\tyellow}[1]{\textcolor{Yellow}{#1}}
\newcommand{\tpurple}[1]{\textcolor{Purple}{#1}}
\newcommand{\tblack}[1]{\textcolor{Black}{#1}}
\newcommand{\tgrey}[1]{\textcolor{Grey}{#1}}
\newcommand{\tdeepred}[1]{\textcolor{deepred}{#1}}
\newcommand{\ttt}[1]{\texttt{#1}}
% MATH -----------------------------------------------------------
\newcommand{\norm}[1]{\left\Vert#1\right\Vert}
\newcommand{\abs}[1]{\left\vert#1\right\vert}
\newcommand{\set}[1]{\left\{#1\right\}}
\newcommand{\Real}{\mathbb R}
\newcommand{\eps}{\varepsilon}
\newcommand{\To}{\longrightarrow}
\newcommand{\BX}{\mathbf{B}(X)}
\newcommand{\A}{\mathcal{A}}
\newcommand{\lv}{\left\langle}
\newcommand{\rv}{\right\rangle}
\newcommand{\mbf}[1]{\mathbf{#1}}
\newcommand{\mat}[2][rrrrrrrrrrrrrrrrrrrrrrrrr]{\left(\begin{array}{#1}#2\\ \end{array}\right)}
% ----------------------------------------------------------------
\setlength{\topmargin}{-.3in}
\setlength{\headheight}{.2in}
\setlength{\headsep}{.3in}
\setlength{\oddsidemargin}{0in}
\setlength{\evensidemargin}{0in}
\setlength{\textwidth}{6.5in}
\setlength{\textheight}{8.5in}
\renewcommand{\baselinestretch}{1.5}
% ----------------------------------------------------------------

\begin{document}

\title{Math 5773: Chapter 3 Summary\\ The Heat Equation}
\author{Troy Butler}

\maketitle


\setcounter{section}{3}
\section*{Chapter 3 -- The purpose}

This chapter presents Fourier's method for solving PDEs and focuses primarily on the heat equation.
Finite difference methods for the heat equation are not considered until chapter 4.
The solution method and the resulting solution, written as an infinite series in the more general cases, are rather formal in the sense that most analysis is left to the later chapters that we will probably not have time to cover in detail in this course.
The approach for solving the heat equation is quite algorithmic and easy to follow even if some of the steps are at times annoying to complete. 
Most of the chapter concerns itself with showing the various steps in the algorithm, but the last two sections of the chapter touch on some general approaches for studying properties of solutions to PDEs that do not require any particular way to represent the solution.

\subsection{A Brief Overview}

A nice overview is given that outlines the typical four steps involved in using either Fourier series or generalized Fourier series\footnote{See Sturm-Liouville Problems in Chapter 8 for more information.}.
We first make an ansatz\footnote{This simply means a rather fortunate guess at the structure of the solutions.} that allows us to separate the variables and identify a family of functions that solve the problem. 
This leads to step 1, but let's skip that for a minute and note that by the linearity of the differential operator, we have that any linear combination of solutions to the homogeneous problem is also a solution. 
With this in mind, Steps 1 and 2 can be combined and rephrased as ``find a basis for the solution space to the homogeneous problem.''
Note that the choice of basis depends upon both the differential equation {\em and} the form of the boundary conditions.
Once a basis for the solution space has been determined, then just like we would expect, solving the particular problem at hand comes down to determining what specific linear combination of the basis produces the solution we seek.
This is what defines step 3.
The fourth step is then to check that everything done in the first three steps is justified and makes sense!
However, we leave this fourth step to later chapters. 
After reading through this chapter, students should create a Python notebook for Project 3.1 which quite systematically steps the students through the process of understanding solutions to a semidiscrete system\footnote{This type of system arises when doing the so-called method of lines discretization, which simply means we discretize in space and treat each ``degree of freedom'' in the spatial domain mesh as a state variable in a first-order system of ODEs. Drawing a picture helps understand this.}. 

\subsection{Separation of Variables}

Inserting the ansatz into the differential equations and then separating the variables produces two ordinary differential equations (with boundary or initial conditions) that we can easily solve to determine the structure of the solution in both space and time. 
Specifically, we identify a useful {\em basis} for the solution space to the heat equation with homogeneous Dirichlet boundary conditions. 
Students should be able to do Exercise 3.6 immediately after reading this section. 

You may notice just how often in textbooks on PDEs the boundary conditions are assumed zero, and this may seem rather restrictive and not general enough to study.
How do you think things would change if the homogeneous Dirichlet conditions were replaced by nonhomogeneous Dirichlet conditions, i.e., what if $u(0,t)=a$ and $u(1,t)=b$ for $t>0$\footnote{The book starts to ask this question in Section 3.5, but I think we should start addressing it now.}?
You may be impetuous and just try to replace $X_k(0)=a$ and $X_k(1)=b$ in Eq.~(3.14) in the text.
Stop, and ask yourself a rather fundamental question
\begin{quote}
Does it even make sense to discuss an eigenspace if the boundary values are nonhomogeneous?
\end{quote}
Recall that an eigenspace is a vector\footnote{The function spaces we are working in are technically vector spaces, so we can refer to the functions as vectors as long as we understand this.} subspace of the vector space\footnote{In this case, the vector space is $C^2([0,1])$.}. 
Using basic linear algebra, we see that any ``set of vectors'' that satisfies the differential equation and nonhomogeneous boundary value problems cannot be a vector subspace since adding any two vectors from this set results in a vector not in the set.


Suppose $v$ is a function satisfying
\begin{eqnarray*}
	v_t &=& v_{xx}, \ \text{ for } x\in(0,1), \ t>0,\\
	v(0,t) &=& a, \ \text{ for } t>0, \\
	v(1,t) &=& b, \ \text{ for } t>0.
\end{eqnarray*}
Now, given $v$, suppose $w$ is a function satisfying
\begin{eqnarray*}
	w_t &=& w_{xx}, \ \text{ for } x\in(0,1), \ t>0,\\
	w(0,t) &=& 0, \ \text{ for } t>0, \\
	w(1,t) &=& 0, \ \text{ for } t>0, \\
	w(x,0) &=& f(x)-v(x,0), \ \text{ for } x\in(0,1).
\end{eqnarray*}
Clearly, $w$ is solving a problem of the type given by Eqs.~(3.1)--(3.3) in the text.
If we then let $u=w+v$, it follows that $u$ solves the problem we want with the nonhomogeneous Dirichlet conditions. 
All we need to do is figure out if an appropriate function $v$ exists. 
Check that $v(x,t)=a+(b-a)x$ is such a function (note that this function is temporally invariant, i.e., it does not depend upon $t$). 
What does this mean? 
It means that while we are developing the approach in the text for solving the problem with homogeneous Dirichlet conditions, we are simultaneously developing an approach for solving the problem with nonhomogeneous Dirichlet conditions. 
Compare this to what the authors are asking you to do in Exercise 3.11. 

The approach shown above of creating two problems where one has homogeneous boundary conditions and the other nonhomogeneous boundary conditions is quite common when dealing with {\em linear} differential operators. 
If we can show the existence of a solution to the nonhomogeneous boundary condition problem, then we lose no generality by just assuming the boundary conditions are homogeneous.
However, it is certainly possible to define boundary conditions in a way for which there are no solutions, and we are sometimes restricted to defining boundary conditions to belong to a certain function space in order for solutions to be guaranteed to exist. 
This is something that is best discussed in the context of trace operators after you have taken some functional analysis. 
Just know that not all boundary conditions are reasonable.

\subsection{The Principle of Superposition}

You may want to look over Exercise 1.16 from Chapter 1 prior to reading this section of Chapter 3.
In part (c) of that exercise, we saw that the linear combination of solutions to the heat equation with different initial data was also a solution to the heat equation with initial data defined by the linear combination of the different initial data. 

In this section, we see that if we can write the initial data as a linear combination of the spatial basis functions (e.g., as either (3.21) or (3.25)), then the solution to the heat equation is immediately given also in terms of a linear combination that is easily written (e.g., as either (3.22) or (3.26), respectively). 
While this is wonderfully simple, it all rests upon the idea that we can somehow write the initial data in a special way.
Example 3.2 shows how why we may require the linear combination to involve an infinite number of terms for ``equality'' to hold.

We say ``equality'' rather loosely here. 
Check out the Python notebook on this chapter that recreates Figure 3.3 for various values of $N$ that also increases the resolution of the mesh used for discretizing $[0,1]$ (you may come to the conclusion that Figure 3.3 is at worst a lie and at best misleading). 
As $N$ increases, the magnitude of the oscillations have a tendency to increase towards $x=0$ and $x=1$ while they decrease towards the middle of the domain\footnote{Of course, the overall amplitude of the oscillations stays bounded.}. 
This is known as Gibbs phenomena.
Also, every single partial sum has the property that it defines a function equal to $0$ at both $x=0$ and $x=1$.
So, how on earth can such a sequence of partial sums converge to a function that is $1$ at $x=0$ and $x=1$?
The answer is that it cannot!
At least, it cannot in the pointwise sense, and most certainly cannot in the $\sup$-norm commonly used on $C([a,b])$ and its subspaces.
The convergence is actually in $L^2$ (in later chapters, the authors choose to refer to this type of convergence as convergence in {\em mean square}). 
Basically, if we integrate the square of the difference of the limit function and the partial sums, then this is a measure of error that goes to zero as we let the number of terms used in the partial sums increase to infinity. 
The main point is this: take measure theory and functional analysis. e in {\em mean square}). 
Basically, if we integrate the square of the difference of the limit function and the partial sums, then this is a measure of error that goes to zero as we let the number of terms used in the partial sums increase to infinity. 
The main point is this: take measure theory and functional analysis. 

Is this pointwise error much of an issue?
Examine the structure of the solution given in (3.24).
Observe the exponential decay in front of each term in the series and how the exponential decay rate {\em increases} as the terms get higher.
It is precisely the higher order terms that cause the pointwise errors near the boundaries.
These errors are going to be ``shrunk in time'' very quickly.
In general, the heat equation is said to ``smooth out'' any rough initial data very quickly.
From a signal processing point of view, ``roughness'' corresponds to high frequency terms being present. 
In a sense, we can think of the heat equation as providing a low-pass filter (i.e., it ``filters out'' the high frequency noise in an initial signal very quickly).  
The structure of the solution shown in (3.24) offers some insight into the mathematical mechanisms that could be used to predict such behavior. 
Play around with this solution in the Python notebook to compare to Figure 3.4.

\subsection{Fourier Coefficients}

Recall from Section 2.4.1 that the eigenfunctions are orthogonal. 
This provides a convenient mechanism for computing the Fourier coefficients of a general function $f(x)$ as shown in (3.29).
Edit the Python notebook to investigate Example 3.4.
At this point, students should be able to complete Exercises 3.1--3.4 without much difficulty. 


\subsection{Other Boundary Conditions}

This is a short and straightforward section.

\subsection{The Neumann Problem}

This simply reproduces the process for Neumann boundary conditions.
It is a good idea for students to use the Python notebook to investigate the previous examples assuming these new initial conditions and remark on any differences. 
After working through this subsection, students should be able to do Exercises 3.5, 3.7--3.11, and 3.14 with little difficulty.

Exercises 3.12 and 3.13 are perhaps more interesting as they involve inhomogeneous equations (i.e., there are nontrivial forcing functions). 
Exercise 3.12 provides a hint that is similar to Exercise 3.11, and in the discussion of Separation of Variables above, students should be able to identify that the essence behind this hint is to define two separate problems that ``sum'' to give the problem at hand.
Exercise 3.13 walks the student through using Fourier series for solving more general inhomogeneous problems. 

Exercise 3.15 basically redoes everything with periodic boundary conditions to help complete the picture of what basis functions look like in general.
We typically refer to the ``full Fourier series'' as that given by the basis functions of both sines and cosines as shown in this problem.

\subsection{Energy Arguments}

Energy arguments are useful for determining certain properties of solutions to PDEs without requiring any knowledge of the structure of an explicit form of the solution.
For example, we can often study bounds or monotonicity (Theorem 3.1) or uniqueness (Corollary 3.1) with energy arguments.

\subsection{Differential of Integrals}

This section simply provides the justification for the techniques used in the previous section on energy arguments.
After reading this and the previous subsection, students should be able to do Exercises 3.16--3.20. 


\end{document}
